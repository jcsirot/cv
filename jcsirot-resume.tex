\documentclass[11pt,a4paper,sans]{moderncv}

\moderncvstyle{classic}
\moderncvcolor{blue}

\usepackage[utf8]{inputenc}
\usepackage[scale=0.80]{geometry}

% personal data
\name{Jean-Christophe}{Sirot}
\title{Software Engineer at Docker}         % optional, remove / comment the line if not wanted
\address{35, rue Bourgelat}{94700 Maisons-Alfort}{France}% optional, remove / comment the line if not wanted; the "postcode city" and "country" arguments can be omitted or provided empty
\phone[mobile]{+33~6~15~16~44~76}                  % optional, remove / comment the line if not wanted; the optional "type" of the phone can be "mobile" (default), "fixed" or "fax"
\email{jc@sirot.org}                               % optional, remove / comment the line if not wanted
\social[twitter]{jcsirot}                          % optional, remove / comment the line if not wanted
\social[linkedin]{jcsirot}                           % optional, remove / comment the line if not wanted
\social[github]{jcsirot}                           % optional, remove / comment the line if not wanted

\begin{document}

\makecvtitle

\section{Who I am}
I am a software development engineer with 15 years of experience and a QA engineer in multiple fields (security, video, advertising). I love building softwares and sharing knowledge and best practices with the development teams to help them make better products.

\section{Work Experience}

\cventry{Oct 2017--Present}{Software Engineer}{Docker}{Paris}{}{%
Software Quality Engineer on the Docker EE / Kubernetes integration
}

\cventry{Feb 2016--Sep 2017}{Lead Developer and Java Expert}{Weborama}{Paris}{}{%
Lead developer on the BigSea project (socio demographic and behavioural profile factory).
Technologies: Java, Apache Storm, Hadoop, Spark, AWS
Main activities:
\begin{itemize}%
    \item Develop features, fix bugs and improve platform architecture
    \item Spread development best practices, automate integration tests, improve coverage
    \item Revamp CI/CD: new maven repository and docker registry, upgrade to CI/CD as code (use Jenkins pipelines and Jenkinsfiles).
\end{itemize}%
}

\cventry{Nov 2014--Feb 2016}{Quality Assurance Manager}{Arkena}{Ivry-sur-Seine}{}{%
Manager of the Arkena QA team: 4 engineers in Paris and 2 engineers Stockholm. I hired 3 of them.
The team was in charge of:
\begin{itemize}%
    \item The validation of the product specifications and the deliverables (CDN, Cloud4Media products, Arkena Video Platform and OTT solutions). Setup of integration and end-to-end tests.
    \item The continuous integration platform and the test automation. Tools: Jenkins, Mesos, Docker, Ansible
    \item The setup and the management of the pre-production platforms
\end{itemize}%
}

\cventry{Dec 2013--Nov 2014}{CDN Quality Assurance Engineer}{Arkena}{Ivry-sur-Seine}{}{%
Design and setup of the Arkena CDN validation platform. Automation of the validation process:
\begin{itemize}%
    \item Installation of the QA platform (20+ hypervisors, multiple VLAN...) and services (Jenkins, Sonarqube, Artifactory, QA Complete)
    \item Build the automatic deployment and validation of CDN components. A dedicated Jenkins plugin has been developed for this purpose
    \item Develop testing tools (Video player simulator, automatic deployment scripts\ldots)
\end{itemize}%
CDN components validation:
\begin{itemize}%
    \item Write and execute test plans (feature tests, load tests, fuzzy tests)
    \item Make proposals in order to improve quality of development
\end{itemize}%
}

\cventry{Apr 2011--Dec 2013}{Build \& Quality Assurance Manager}{Cryptolog}{Paris}{}{%
Lead of the quality assurance team.\newline{}%
Build Management: management of the build environments (Windows / Linux / Mac OS X) and tools for all development teams.\newline{}%
Quality Assurance: Development and software quality guides. Quality metrics analysis. Automated testing.\newline{}%
}

\cventry{Jan 2005--Apr 2011}{Project Manager}{Cryptolog}{Paris}{}{%
Technical lead of digital signature product suite. Managing the development team (2-4 people).%
}

\cventry{Oct 2002--Jan 2005}{Security Software Engineer}{Cryptolog}{Paris}{}{%
Development of libraries and servers applications in the field of digital signature and Public Key Infrastructure. Most develoment was made with Java and C.\newline{}%
}

\section{Extra activities}

\cventry{May 2016-Today}{Co-Founder}{Association Cloud Native France (CNCF Paris)}{Paris}{}{%
We created a non-profit organisation to promote Cloud Native concepts and the Cloud Native Computing Foundation (\url{https://cncf.io}). We organize one meetup per month \url{https://meetup.com/fr-FR/Cloud-Native-Computing-Paris/}%
}

\cventry{Devoxx France 2015}{Speaker}{De la cryptographie dans le navigateur avec WebCrypto API}{}{}{%
\url{http://fr.slideshare.net/JeanChristopheSirot/web-cryptoapi-devoxxfr2015}
}

\cventry{Devoxx France 2016}{Speaker}{Hands on lab: Déployez vos applications sur un cluster Kubernetes avec Ansible}{}{}{}

\cventry{Ansible meetup Apr 2016}{Speaker}{Ansible et Jenkins}{}{}{%
\url{https://www.infoq.com/fr/presentations/ansible-jean-christophe-sirot-ansible-et-jenkins}
}

\cventry{Devoxx France 2017}{Speaker}{Tools in action: Raconte-moi X.509 : anatomie d'une autorité de certification}{}{}{}

\cventry{Today}{Open source developer}{}{}{}{%
\begin{itemize}%
    \item \textit{Jenkins Ansible plugin} \url{https://wiki.jenkins-ci.org/display/JENKINS/Ansible+Plugin}
    \item \textit{Jenkins Custom Job Icon plugin} \url{https://wiki.jenkins-ci.org/display/JENKINS/Custom+Job+Icon+Plugin}
    \item \textit{digest.js} a cryptographic digest javascript library \url{https://github.com/jcsirot/digest.js}
\end{itemize}%
}

\section{Education}

\cventry{2001--2002}{M.Sc. in Electronic Commerce}{Dublin City University}{Dublin - Ireland}{}{Specializing in Cryptography. Jointly delivered by the DCU School of Computing and the DCU Business School}

\cventry{1999--2002}{M.Sc. in Computer Science}{Telecom Nancy}{Nancy - France}{}{Specializing in software development}

\section{Languages}
\cvitemwithcomment{French}{Native}{}
\cvitemwithcomment{English}{Full Professional Proficiency}{}
\cvitemwithcomment{German}{Limited Working Proficiency}{}


\section{Miscellaneous}

\cvlistitem{Massive Open Online Courses participant (IT, Mathematics, Physics...)}
\cvlistitem{Video game and board game player}

\end{document}
