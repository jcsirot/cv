\documentclass[11pt,a4paper,sans]{moderncv}

\usepackage[utf8]{inputenc}
\usepackage[scale=0.80]{geometry}
\usepackage{makecell}

%\makeatletter
%\patchcmd{\makehead}{%search
%    \setlength{\makeheaddetailswidth}{0.8\textwidth}%
%    }{%replace
%    \setlength{\makeheaddetailswidth}{\textwidth}%
%  }{%success
%  }{%failure
%  }
%\makeatother

\moderncvtheme{banking}
\moderncvcolor{green}

% personal data
\firstname{Jean-Christophe}
\familyname{Sirot}
\title{Staff Engineer}                   % optional, remove / comment the line if not wanted
\address{35, rue Bourgelat}{94700 Maisons-Alfort}{France}% optional, remove / comment the line if not wanted; the "postcode city" and "country" arguments can be omitted or provided empty
\phone[mobile]{+33~6~15~16~44~76}                  % optional, remove / comment the line if not wanted; the optional "type" of the phone can be "mobile" (default), "fixed" or "fax"
\email{jc@sirot.org}                               % optional, remove / comment the line if not wanted
\social[twitter]{jcsirot}                          % optional, remove / comment the line if not wanted
\social[linkedin]{jcsirot}                         % optional, remove / comment the line if not wanted
\social[github]{jcsirot}                           % optional, remove / comment the line if not wanted

\begin{document}

\makecvtitle

\section{Who I am}
I am a senior software development engineer with 20 years of experience in various domains, mainly digital signature, Public Key Infrastructure and Cloud Native. I love designing softwares, sharing knowledge and best practices with the development teams to help them build better products.

\section{Work Experience}

\subsection{\httplink[Aircall]{aircall.io}, Paris}

\cventry{Dec 2021--Present}{Staff Engineer}{}{}{}{%
Creation of a Developer Productivity team at Aircall.
\begin{itemize}%
    \item Develop and promote internal tools along with best practices for the backend teams (AWS / Serverless micro-services / typescript)
    \item Collaborate with the architecture team to define the backend reference architecture
    \item Implement the reference architecture in sample and demo projects
\end{itemize}
}

\cventry{Mai 2020--Dec 2021}{Lead Software Engineer}{}{}{}{%
Led the Authentication team of 7 people. The team was responsible for:
\begin{itemize}%
    \item Extracting the authentication module from a Rails monolith, developing and deploying the new authentication stack
    \item Ensuring seamless migration of the customer authentication database
\end{itemize}
\textit{Technologies: AWS Cognito, AWS Lambda, Google Sign-In, SAML}
}

\vspace*{6mm}

\subsection{\httplink[Docker]{www.docker.com}, Paris}

\cventry{Jan 2019--Jan 2020}{Software Engineer}{}{}{}{%
As a Software Engineer in the Developer Solution Group, I was responsible for the development and maintenance of two projects:
\begin{itemize}%
    \item \httplink[Docker App]{https://github.com/docker/app}, a packaging and deployment application for cloud-native applications. My tasks included developing new features and adapting it to the CNAB standard (\url{https://cnab.io}).
    \item \httplink[Docker Compose]{https://github.com/docker/compose}. I triaged tickets and fixed bugs, and also rewrote and fully automated the release process.
\end{itemize}%
}

\cventry{Oct 2017--Jan 2019}{Software Engineer}{}{}{}{%
Software Quality Engineer on the Docker EE / Kubernetes integration.
\begin{itemize}%
    \item Create a test platform for the Docker Enterprise embedded Kubernetes (AWS, Azure, vSphere)
    \item Automatic reporting of test results and failures with the development of an integration plugin for Testrail
    \item Run the Kubernetes E2E test suite and write patchs when required
\end{itemize}%
\emph{Technologies: Docker, Kubernetes, Go, Jenkins, Ansible}
}

\vspace*{10mm}

\subsection{\httplink[Weborama]{www.weborama.com/}, Paris}

\cventry{Feb 2016--Sep 2017}{Lead Developer and Java Expert}{}{}{}{%
As the lead developer of the BigSea team, I was responsible for the migration of the \textit{Proof of Concept} product from Apache Storm to a production-ready Kubernetes-based application that creates sociological, demographic, and behavioral profiles.
\begin{itemize}%
    \item Developed and implemented new features using Java, Hadoop, and Spark
    \item Successfully migrated the application to a Kubernetes cluster running on AWS EC2 instances
\end{itemize}%    
}

\vspace*{6mm}

\subsection{\httplink[Arkena]{www.cognacqjayimage.com}, Ivry-sur-Seine}

\cventry{Nov 2014--Feb 2016}{Quality Assurance Manager}{}{}{}{%
Manager of the Arkena QA team: 6 engineers in Paris and Stockholm.%
}

\cventry{Dec 2013--Nov 2014}{CDN Quality Assurance Engineer}{}{}{}{%
Design and setup the new Arkena CDN validation platform. Automation of the CDN components validation process.%
\emph{Technologies: Jenkins, Docker, Ansible, Sonarqube, Artifactory}%
}

\vspace*{6mm}

\subsection{\httplink[Universign]{www.universign.eu}, Paris}

\cventry{Apr 2011--Dec 2013}{Build \& Quality Assurance Manager}{}{}{}{%
Lead of the quality assurance team. Management of the build environments (Windows, Linux, OSX) and tools for the dev teams. Development of the software quality guides and quality metrics analysis.
}

\cventry{Jan 2005--Apr 2011}{Tech Lead}{}{}{}{%
Technical lead of digital signature product suite. Manager of a development team of 4 people.%
}

\cventry{Oct 2002--Jan 2005}{Software Engineer}{}{}{}{%
Development of libraries and servers applications in the digital signature and Public Key Infrastructure fields. Most develoment was made with Java and C.\newline{}%
}

\section{Extra activities}

\cventry{2022-Present}{Event Organizer}{Kubernetes Community Days France 2023}{}{}{%
Creation of Kubernetes Community Days France 2023, a conference focused on Cloud-Native and DevOps, sponsored by the Cloud Native Computing Foundation, which took place on March 7th, 2023.\newline{}%
\url{https://kcdfrance.fr}%
}

\cventry{2016-Present}{Co-Founder}{Association Cloud Native Computing Paris (CNCF Paris)}{}{}{%
We created a non-profit organisation to promote Cloud Native technologies and the \httplink[Cloud Native Computing Foundation]{https://cncf.io}.\newline{}%
\url{https://meetup.com/fr-FR/Cloud-Native-Computing-Paris/}%
}

\cventry{2015-Present}{Team Member}{Paris Java User Group}{}{}{%
Member of the team organizing meetups every month \url{https://www.parisjug.org/}%
}

\section{Speaker experience}

\cventry{Devoxx France 2022}{Speaker}{Conference: La Cryptanalyse d'Enigma : entre espionnage et mathématiques}{}{}{%
\begin{itemize}%
    \item \emph{Slides} \url{https://bit.ly/3CjArC2}
    \item \emph{Video} \url{https://youtu.be/TnxPRKPpEoA}
\end{itemize}
}

\cventry{Devoxx France 2019}{Speaker}{Université: Cycle de vie des applications dans Kubernetes}{}{}{%
\emph{Video} \url{https://youtu.be/-PrLE3FD6mw}%
}

\cventry{Devoxx France 2018 / Devoxx Maroc 2018}{Speaker}{Tools in Action: Java dans Docker Bonnes pratiques}{}{}{%
\begin{itemize}%
    \item \emph{Slides} \url{https://fr.slideshare.net/kanedafromparis/java-indocker-devoxxfr-2018}
    \item \emph{Video} \url{https://youtu.be/vzpU2jxrxJ8}
\end{itemize}
}

\cventry{Ansible meetup Apr 2016}{Speaker}{Ansible et Jenkins}{}{}{%
\emph{Video} \url{https://bit.ly/3CfjOaW}
}

\cventry{Devoxx France 2015}{Speaker}{Lighting Talk: De la cryptographie dans le navigateur avec WebCrypto API}{}{}{%
\emph{Slides} \url{http://fr.slideshare.net/JeanChristopheSirot/web-cryptoapi-devoxxfr2015}
}

\section{Open source contributions}

\cvitemwithcomment{anki-simple-furigana}{\makecell[tl]{an Anki Add-On for students learning Japanese \\ \url{https://ankiweb.net/shared/info/1444055400}}}{}
\cvitemwithcomment{go-polly-tts}{\makecell[tl]{a simple Japanese Text-to-speech generator using AWS Polly \\ \url{https://github.com/jcsirot/go-polly-tts}}}{}
\cvitemwithcomment{Jenkins Ansible plugin}{\url{https://wiki.jenkins-ci.org/display/JENKINS/Ansible+Plugin}}{}
    %\item \textit{digest.js} a cryptographic digest javascript library \url{https://github.com/jcsirot/digest.js}{}

\section{Education}

\cventry{2001--2002}{M.Sc. in Electronic Commerce}{Dublin City University}{Dublin - Ireland}{}{Specializing in Cryptography. Jointly delivered by the DCU School of Computing and the DCU Business School}

\cventry{1999--2002}{M.Sc. in Computer Science}{Telecom Nancy}{Nancy - France}{}{Specializing in software development}

\section{Languages}
\cvitemwithcomment{French}{Native}{}
\cvitemwithcomment{English}{Full Professional Proficiency}{}
\cvitemwithcomment{German}{Limited Working Proficiency}{}
\cvitemwithcomment{Japanese}{Beginner (JLPT N4 level)}{}

\section{Miscellaneous}

\cvlistitem{Learning Japanese at INALCO}
\cvlistitem{Playing Video games and board games}

\end{document}
